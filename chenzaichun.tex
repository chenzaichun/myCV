%!TEX TS-program = xelatex
\documentclass[]{friggeri-cv}
\usepackage{afterpage}
\usepackage{hyperref}
\usepackage{color}
\usepackage{xcolor}
\usepackage{xeCJK}
\setCJKmainfont[BoldFont={Heiti SC Medium}]{Heiti SC}
\setCJKmonofont{Songti SC}
\setCJKsansfont{Heiti SC}
 \newCJKfontfamily[cuhei]\cuti{Heiti SC Medium}
\hypersetup{
    pdftitle={},
    pdfauthor={},
    pdfsubject={},
    pdfkeywords={},
    colorlinks=false,       % no lik border color
   allbordercolors=white    % white border color for all
}
\addbibresource{bibliography.bib}
\RequirePackage{xcolor}
\definecolor{pblue}{HTML}{0395DE}

\begin{document}
\header{陈载春}{}
      {}
      
% Fake text to add separator      
\fcolorbox{white}{gray}{\parbox{\dimexpr\textwidth-2\fboxsep-2\fboxrule}{%
.....
}}

% In the aside, each new line forces a line break
\begin{aside}
  \section{\cuti 地址}
    重庆市沙坪坝区
  \section{\cuti 电话~ QQ~ 微信}
    +86 136 3789 4552
    160 152 79
    czc112358
  \section{Mail}
    \href{mailto:chenzaichun@gmail.com}{\textbf{chenzaichun@}\\gmail.com}
  \section{Web \& Git}
    \href{http://yesokay.herokuapp.com}{yesokay.herokuapp.com}
    \href{https://github.com/chenzaichun}{github.com/chenzaichun}
  \section{\cuti 编程技能}
    ~
    \includegraphics[scale=0.62]{img/programming.pdf}
    ~
  \section{\cuti 操作系统运维}
    \textbf{GNU/Linux}\includegraphics[scale=0.16]{img/5stars}
    \textbf{MacOS}\includegraphics[scale=0.16]{img/4stars}
    \textbf{Windows}\includegraphics[scale=0.16]{img/4stars}
    \textbf{Android}\includegraphics[scale=0.16]{img/3stars}
    \textbf{iOS}\includegraphics[scale=0.16]{img/3stars}
    \textbf{HPUX}\includegraphics[scale=0.16]{img/2stars}
    \textbf{Solaris}\includegraphics[scale=0.16]{img/2stars}
  \section{\cuti 其他技能}
    ~
    \includegraphics[scale=0.62]{img/personal.pdf}
    ~
  \section{\cuti 语言}
    \textbf{English}\includegraphics[scale=0.16]{img/3stars}
\end{aside}

\section{\cuti 重要项目经历}
\begin{entrylist}
  \entry
    {2011.4 - 至今}
    {\cuti TeMIP Trouble Ticket Liaison}
    {HP}
    {项目主要是给电信厂商提供设备告警和 Trouble Ticket 的管理,并和其他 ITSM 系统集成。\\}

    \entry
    {2014.11 - 2015.2}
    {\cuti Vision Media Management}
    {HP}
    {Vision Media Management 是美国一家电影道具管理和租赁公司。我在项⺫中 负责该公司 iOS 项⺫的开发。该 App 主要是提供给他们的销售人员用来展示 电影中所使用的道具。\\}

    \entry
    {2010.10 - 2011.4}
    {\cuti GTS V2}
    {HP}
    {该项目主要是给美国通用提供⻋辆自动化测试的工具。我在项目中负责升级 新需求并发布新版本。\\}
    
    \entry
    {2018.12 - 2010.9}
    {\cuti 东方 Online}
    {}
    {3D 武侠⻛格 MMORPG,在项⺫中主要负责服务器端和客户端的开发。 \\}

    \entry
    {2008.1 - 2008.10}
    {\cuti 星球计划}
    {宏信}
    {3D 科幻⻛格 MMORPG,在项目主要负责客户端逻辑和 UI 的开发。\\}

    \entry
    {2006.7 - 2017.12}
    {\cuti 桑德智能⻋载系统}
    {桑德}
    {主要负责桑德⻋载智能系统的开发,包括主动防御,GPS 定位,多媒体系统 和移动办公。\\}
\end{entrylist}

\section{\cuti 高等教育经历}
\begin{entrylist}
  \entry
    {2002 - 2006}
    {\cuti 软件工程 学士学位}
    {重庆大学软件学院}
    {\\}
  \\}
\end{entrylist}

\vspace{40pt}

\begin{flushright}
\emph{二〇一七年二月五日}
\end{flushright}
\begin{flushright}
\emph{陈载春}
\end{flushright}

%%% This piece of code has been commented by Karol Kozioł due to biblatex errors. 
% 
%\printbibsection{article}{article in peer-reviewed journal}
%\begin{refsection}
%  \nocite{*}
%  \printbibliography[sorting=chronological, type=inproceedings, title={international peer-reviewed conferences/proceedings}, notkeyword={france}, heading=subbibliography]
%\end{refsection}
%\begin{refsection}
%  \nocite{*}
%  \printbibliography[sorting=chronological, type=inproceedings, title={local peer-reviewed conferences/proceedings}, keyword={france}, heading=subbibliography]
%\end{refsection}
%\printbibsection{misc}{other publications}
%\printbibsection{report}{research reports}

\end{document}
